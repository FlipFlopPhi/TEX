\chapter{Introduction}\label{introduction}
The internet is a growing source of revenue and information. Being able to analyze this collection of data is very useful. Most website analysis is done by looking at the content of a website, but not at the website layout itself. However being able to analyze and compare website layouts might be more worthwhile then we think. We could for example remove redundant backups of websites by looking if their layout has changed or use clustering to group up different websites with similar designs.\\
To be able to use modern analysis tools, we have to be able to quantify distance between two elements. If we're able to calculate the distance between different websites we can see which websites are similar and which ones differ the most. A simple solution would be retrieving metrics about the websites design, for example most used color, however the outcome is entirely dependent on the chosen metrics where chosen, and no clear guidelines on constructing such guidelines exist. Unfortunately, when quantifying distance over large and complex data elements, it becomes more difficult to find a purely objective metric. Because, while we'll usually still be able to say whether two websites differ, being able to say what website is more different becomes more of a subjective decision.\\
Because of this, we will be looking at adapting an already existing technique instead of coming up and testing a new one. In our research we will be using the edit tree distance, while it is not known if this objective approach of measuring the distance between layouts is consistent with human judgement, we will assume it is sufficiently close to be meaningful. A technique that can be used to calculate distance between two data trees.\\
To do this, we will have to find a way of translating websites into data trees, so that we can then use our edit-tree-distance to analyze our data.