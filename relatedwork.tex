\chapter{Related Work}\label{relatedwork}
Hier zal ik kijken naar vergelijkbaar onderzoek.

\section{Preliminary knowledge}
Before we dive deeper into our research, it might be useful to explain certain aspects of the information-domains we are using.
\\ A web page's structure is defined by \texttt{div}'s. A \texttt{div} is basically a box in which content (like text or an image) can be displayed. A div can also contain another div which again can contain more content. When the web page is loaded into a browser, the page's \texttt{div}'s and its content are structured into a Document Object Model (DOM). Since the DOM is a data tree we can easily translate it into a tree, on which we can use our edit tree distance algorithm.
\\ The elements of the websites we will be studying, have style properties assigned to them, either using a style-sheet or using inline attributes. These rules explain how they should be displayed, and this is what we will be using to calculate the attributes we want to be stored.
