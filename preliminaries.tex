\chapter{Related work}\label{relatedwork}
Hier zal ik kijken naar vergelijkbaar onderzoek.

\section{Preliminary knowledge}
Before we start diving deeper into our research, it might be useful to explain certain aspects of the information-domains we're using.
A websites layout is defined by \texttt{div}'s. A \texttt{div} is basically a box in which content (like text or an image) can be displayed. A div can also contain another div which again can contain more content. This structure is very similar to a tree; every node (\texttt{div}) of the tree might have one or more leaves (pieces of content) attached to it or might have another node attached to itself. For this research we will convert websites into data-trees so we can calculate the edit distance between them.\\
The elements of the websites we will be studying, have style properties assigned to them, either using a style-sheet or using inline attributes. These rules explain how they should be displayed, and this is what we'll be using to calculate the attributes we want to be stored.
