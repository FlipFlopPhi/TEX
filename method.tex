\chapter{Method}\label{method}

\section{Transformation}
The DOM model contains a lot of information, however a large part of this information is not needed. The DOM also does not directly contain all the information we need. Because of this, we need to be able to transform the DOM into a data tree that contains only the information we need.
\\The first component is the part of the program that is responsible for transforming websites into data elements. This is done by opening a virtual browser window using the Ghost.py library for python. The window we have use, has a resolution of $1920x1080$; while different resolutions can be used when needed, we have chosen to use this resolution, because it is on of the most used resolution on newer computers.
\subsection{Style sheet handling}
Once the window is generated, a website can be loaded in. On every website we use two pieces of JavaScript code. The first, "pre.js" is used to make XMLHttpRequests to load in style-sheets from external domains, and turning them into style-nodes. This is done because because the Same-origin policy does not not allow scripts to acces data from a domain different from the one it's executed at.
\\However it worth noting that, this is only possible if the external domain has CORS enabled. Thus the problem could arise that it is no longer possible to correctly retrieve the css-information via the current method.
\\When all external style-sheets have been converted "core.js", the second piece of JavaScript, is executed. First all the internal style-sheet rules are assigned to their corresponding elements, by making them inline. This is done so that we do not have to look through all the css-rules every-time we want to look up one of the style properties of an element.

\subsection{Tree representation}
 When all of the style attributes have been made in-line, we only have to concern ourselves with the inline style attributes of each element of the DOM tree.
\\Ffr every element we want to keep the following information:
\begin{itemize}
	\item the elements width
	\item the elements height
	\item the elements horizontal offset, relative to its parent
	\item the elements vertical offset, relative to its parent
	\item the children (and their values) belonging to this element
\end{itemize}
We do this by recursively calculating the width and height of every element, and their placement relative to their parent element. We also have an extra slot we use to save information that we can use for debugging. This value is however never used later.
\\When calculating these values another problem arises. Auto generated values that are created when the 'auto' keyword is used, are not always correct. Since the specific implementation of this is browser specific, the exact result depends on the browser that is used. Because we are using a custom browser, these values are not always computed as we want. For example \textit{'margin auto'} does not center elements in our browser, while it would do so in others. Because of this we do not use the \textit{'Window.getComputedStyle()'} method, but we calculate the values ourselves. 

\\Once calculated, the elements and their values are saved to our local database, so that they can be used for analysis.

\section{the analyzer}
The second component of the program can then be used to analyze the new data elements using our own adaption of the edit-tree distance metric.%
\footnote{The adaption of the Edit-Tree-Distance algorithm can be found in the "LooseEditComparer.java" class.}
When we look at an element in our tree, we can simplify it to having two properties, the four values (width, height, vertical offset and horizontal offset) and its possible children. The way we calculate the distance between two branches is by looking at these two properties. We can project the four values to two vectors, and then calculate the normalized distance between the original vector and the vector of the other branch, this gives us the cost of substituting the original element with the other one.\\
Then we calculate the distance between all the children of our element and all of the children of the other element. We use the shortest distance found, or use the cost of creating the specific child if it is smaller than the shortest distance. The distance between two branches is than the total of these two costs.\\
For calculating the average we do a similar thing. Since the four values are actually vectors we can calculate these averages via conventional mathematical means. For the children, we try to find the closest pairs again, and then calculate the averages of those pairs.
